\documentclass[12pt,a4paper]{article}
\usepackage[czech]{babel}
\usepackage[utf8]{inputenc}
\usepackage[colorlinks=true]{hyperref} % Aktivní odkazy
\usepackage{graphicx} %vkládání obrázků
\usepackage{authblk} %Formátování autorů
\usepackage[backend=biber, style=iso-numeric, giveninits=true, maxbibnames=99]{biblatex} %spravuje interakci s literaturou, nastavuje ISO 690

\DeclareNameAlias{default}{last-first} %Formátování jmen v citacích
\makeatletter
\renewcommand\Authands{, } % Formátování autorů pro češtinu
\makeatother

\textwidth 15,92cm \textheight 24.62cm
\topmargin -2.2cm 
\oddsidemargin 0cm \evensidemargin 0cm

\pagestyle{empty}
\addbibresource{refs.bib}
\nocite{*} % Vytiskne celý .bib soubor bez nutnosti citovat

\title{Dálkové měření vzdálenosti pomocí laserového paprsku (LIDAR)}


\author[1]{Jakub Skalka}
\author[2]{S. Š. Žák}
\author[3]{S. Š. Žák }

\date{\small O. Garant, školitel; KDM MFF UK\vspace{-2em}} % It is what it is. Je 16. 6. 9:00, je to šité horkou jehlou. \vspace je důležitý, jinak si budete krást místo pro příspěvek. Vždy musí být u posledního autora

\affil[1]{Gymnázium, České Budějovice, Jírovcova 8; skalkaj@jirovcovka.net}
\affil[2]{G Budějovická, Praha; email@server.cz}
\affil[3]{Wichterlovo G, Ostrava; email@server.cz \vspace{-1em}} %Ditto, \vspace musí být u posledního autora

\begin{document}

\maketitle \thispagestyle{empty}

\begin{abstract} \noindent
Toto je abstrakt, zahrnuje, čemu se MP věnuje, a pokud lze stručně zmínit nějaký výdobytek výzkumu, je vhodno jej zmínit.
\end{abstract}


\section{Úvod}
Úvodní a základní informace (motivace, současný stav problému, teoretický předpoklad)


\section{Tělo příspěvku}
Tady zapíšete, co jste dělali. Prosím, nezapomeňte měnit i nadpisy kapitol (tedy neponechat „Tělo příspěvku“). Klidně si jej rozdělte do více kapitol v příspěvku vlastním. Rozhodně by tato sekce ale měla obsahovat
\begin{enumerate}

\item Materiály a metody
\item Výsledky a jejich diskuse
\end{enumerate}


\section{Shrnutí}
Závěrečné informace, shrnutí, vypíchnutí hledání, zodpovězení otázek, které motivovaly zkoumání, výhled do budoucna.



\section*{Poděkování}
Chtěli bychom velice poděkovat Ing. Kryštofovi Kadlecovi za jeho pomoc a podporu při vytváření tohoto příspěvku.
Náš dík patří také FJFI za umožnění výzkumu v rámci projektu.
\printbibliography

\end{document}